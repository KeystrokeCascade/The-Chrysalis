I suppose if I have to start somewhere, it would be with the day I was reborn.

{\br}%
The first thing I knew was hunger.

Not the kind of hunger that sends you to the pantry to find a quick snack. Not even the kind of hunger that slowly wears you down as you starve to death. I mean the kind of hunger that tears at your very soul, consuming your every thought.

My world was hunger; everything else was a mere flicker of light in the background.

Then my hunger-deranged mind caught the one thing out there that mattered to it: love. It latched onto the intoxicating sensation, pushing me toward a single, simple objective:

\textit{Feed!}

I surged forward. In the distance, past the hunger, I could feel a body—my body!—move in response. It felt so remote, so detached from the hunger and sweet sense of love that had become my entire world.

I pushed forward, and something resisted me. Numbly, instinctively, I lowered my head and pushed again. I was only vaguely aware of the cool sensation that met my horn as I drew a little closer to my prize. Clumsy limbs pushed and flailed as the sensation spread, and I flowed forward. I felt pin-pricks all along my body as everything twisted and turned with my movement, the world momentarily losing all sense of cohesion, then solidifying abruptly as I struck something hard. Past the laser-like focus of my hunger-addled mind, I only barely recognized that it was the ground.

Thick fluid coursed over my body as I reoriented on the love I sensed. My hooves wobbled and slid against the hard ground as I coughed up copious amounts of fluid, but I ignored the difficulties of my body. I scrambled blindly forward, over inconsequential obstacles, until I fell upon the source of the tantalizing emotion.

I didn’t know exactly what I grasped in my hooves. It was small, hard, and most importantly, filled with love.

I fed.

I drew in every little bit of love, and the terrible hunger relaxed its grip on my mind.

Devoid of the powerful, all-consuming drive, my head sunk to the wet ground, feeling so incredibly heavy. Then everything was gone once again.

{\br}%
To this day, I still can’t remember the dream that had followed. The only thing I remember was the voice that had been at the center of my world. My Queen. Queen Ephema.

I don’t remember what she said in my dream. I just remember that she was there, speaking, and that her voice was happy and kind. I wish I could remember more.

I like to think that the dream was of the day I had left the hive for my first solo assignment. The day I had truly stepped up to embrace the mantle of Infiltrator. The day that I could finally return all the effort she and the rest of the hive had invested in me. Queen Ephema had spoken to me many times before, as she did with all of her children, but that day was special. That day, she told me how proud she was of me; how happy she was for me.

I felt as light as a feather at her kind words, a happiness that I clung to during my long months away from the hive.

I don’t know for sure if that’s what I dreamed of, but I like to think it was.

It’s how I want to remember her.

{\br}%
Sound was the first sense to creep its way into my reluctantly conscious mind. I slowly became aware of a quiet, repetitive squeaking nearby. Somewhere in the distance, a klaxon sounded, muffled and echoing, mixed with indistinct words from some far-away speaker. As consciousness returned, I began to notice the soft hum of electronics all around me, and the steady dripping of liquid.

A faint glow caught my attention. A ghostly flicker of orange barely seen through my eyelids, just on the edge of my perception. After a few moments of drowsy contemplation, I finally opened my eyes. A wavering orange splash of light weaved back-and-forth around a blob of green light. It took several slow blinks before my vision started to focus. The wavering orange light was cast by an emergency light. It squeaked as it spun in place, casting its rotating light across a bare concrete wall and filling the dark room with deep, flickering shadows. The green I saw was the light shining through the deflated remains of the chrysalis I had just escaped.

Now, I want to clarify for any who might be confused: that’s \textit{lower}-case chrysalis, a changeling cocoon, not \textit{upper}-case Chrysalis, the changeling queen. I figure it’s probably clear from context, but I wouldn’t want anyone to think I came from \textit{her}.

After contemplating the limp, dripping remains of the chrysalis, I raised my head. Thick fluids dribbled from my chin as I took in more of the room. The first detail I caught stopped me cold. On either side of the cocoon I had emerged from hung many more. The first one I looked at was still full, though the fluid was dark and murky. The spinning light cast shadows across the empty husk of a dead changeling floating within it.

A faint flicker of adrenaline, cold and acidic, began to chase away the remaining sleepiness.

The husk sat still, suspended in the dark muck of its own decay. Vacant eyes stared out, the empty carapace tangled up in wires that attached to it in several places. I stared numbly at it for several seconds before I swallowed—despite the fluids still slowly dripping from me, my mouth felt dry—and looked back to the chrysalis I had escaped. Similar wires hung from the gash my horn had torn in the clear membrane, the pads that had attached them to my body dangling.

I looked slowly over the line of cocoons, ten in total. Only one other was still full, and its occupant had fared no better. The rest were torn and deflated, dangling in tattered ribbons from the pipes above. A couple more husks hung in mid-air, entangled in the remains of the chrysalises that had held them. Others had fallen to the ground. Some, I realized in horror, had survived their internment only to die shortly after gaining their freedom. One of the obstacles I had blindly stumbled past was the dead husk of another changeling, sprawled out on the floor as if it had been trying to reach the same life-giving store of love I had just devoured. The empty eyes of the exoskeleton stared in my direction, as if blaming me for her death.

I lay my head down, eyes shutting again. My heart rate and breathing had accelerated, and I focused on breathing slowly and deeply. I had to focus and stay calm. I had a responsibility. My hive had invested years of training, preparing me to deal with stressful and dangerous situations. I hadn’t anticipated ever finding myself in such a horrible situation, but I knew I couldn’t let that stop me. I owed all of them that much and more.

Especially the ones who lay dead, where I had lived.

So I wrestled down the feeling of fear and despair that clawed at the back of my mind. I slowly relaxed, stilling the trembling that sought to take over my body. I lay there in the room as I focused on the quiet, repetitive noises all around me.

I’m not sure how long I simply lay there in that shallow puddle of fluids. I knew I couldn’t stay there forever, that I had to eventually get up and deal with whatever had happened. That was slightly complicated by having no idea what \textit{had} happened. My mind was still slow and groggy from the induced hibernation within the chrysalis, and my memory was spotty at best. I needed to get up, figure out what had happened, and find my way from there.

But still, I just lay there, ignoring the world around me, as if by refusing to acknowledge it, it might cease to exist. I ignored the empty husks of my dead siblings. I ignored the uncomfortable object beneath my chest, which I had collapsed atop after my meager feast. I ignored the form that lay beside me, glimpsed in the corner of my eye; a large form, still and silent, looming over me.

I didn’t want to look. I wanted to lay there, as if I could pretend there was nothing wrong. It was tempting, but I knew I couldn’t. I owed it to my hive, my Queen. I had to look.

I had to face the truth, no matter how it hurt.

So I opened my eyes again, shakily raising my head, and turned to look at the empty husk of Queen Ephema. Even in death, she still looked regal, laid out with her head leaning against the concrete wall.

I trembled, giving a weak sob around the tightness in my throat. I might have even cried a little. Part of me wanted to just give up; to curl up beside her, close my eyes, and join her. My Queen was dead. My hive, likely, as well. Why else would she still be lying here, uncared for? It was a disgrace, a defilement. Noling would have permitted this, had they been able to do anything about it.

Another dead changeling lay beside her, a foreleg draped across one of the Queen’s. The gaping eyes showed me—in more detail than I could have ever hoped or wanted—the emptiness that lay within the exoskeleton. Everything that had made them who they were had decayed away long before I had woken, leaving nothing but empty husks behind.

I looked to the other changeling lying beside the Queen, and wondered if she had given up, preferring death at her Queen’s side over carrying on with such a horrible burden. I quickly cast the thought aside. Whoever she had been, she deserved better than such dark thoughts.

A sudden fit of coughing wracked my body, wet and phlegmy. I managed to cough up another mouthful of liquid, spitting it to the floor.

Recovering from the coughing fit, I suddenly realized the lump under my chest was the Queen’s other foreleg. Ignoring the weakness in my limbs, I quickly scrambled to my hooves. It wasn’t out of disgust or fear from contacting a… a corpse. It simply seemed disrespectful to treat her remains as if they were a common cushion.

As I stood, a flicker of light and a quiet clatter drew my attention down to the concrete floor by my hooves. A crystal the size of one of my fangs lay there, reflecting the spinning light. I knew immediately what it was; a love crystal, a wonderful little device for storing, as you might guess, love. A small one like it could hold enough love to keep a changeling fed for weeks, but this one lay empty. I had consumed what little love was left in it.

Several more lay encircled by my Queen’s forelegs, as if she were protecting them. All were empty, without even the faintest trace of love remaining. Had they been consumed? Or had their love leaked away, like a spoiling apple? How long could those crystals hold love?

I saw it as an offering; one final gift to her children. I didn’t doubt it for a second. She lay there as if she knew what was coming, and still she sought to provide for us. I shuddered, eyes tearing up again. Her gift was the only reason I was alive.

Even with that gift, my hunger still had its claws sunk in deep. I needed to find a source of love soon, or her gift would be wasted. A few days, maybe. Even refraining from the use of magic, I wouldn’t last a week.

Only then did I notice the object held between her hooves. I stared at it through blurry eyes for several seconds before I realized what it was.

A data-store. Small, sturdy, able to hold tremendous amounts of digital data. And she held it, as if offering it to me.

I lurched forward only to halt myself. Then, slowly and reverently, I took the data-store gently in my hooves and lifted it away, a few droplets of water falling from the device. I held it like some holy relic, staring at it as if I could read its contents through my rapt attention. Whatever was in there, my Queen had held on to it, protecting it even more than the crystals she had left for us. A message for us, her daughters, who had survived her. I needed to know what secrets it held.

Which meant I needed a computer.

I wiped a hoof across my eyes—managing only to smear around the fluids clinging to my carapace—and looked about, finally taking in the room around me.

The chrysalises filled the center of the room, hung from the exposed piping and arranged in a semi-circle around a slightly raised central platform. Machinery was arranged behind and beside each one, with cables and tubes leading to the various cocoons—or, in many cases, splayed out across the ground where they had fallen upon their cocoon’s failing. Most of those loose cables lay in puddles, and not all of it came from the dripping, just-vacated chrysalis. The walls were stained from the trickling of water running over them, and drops fell from some of the pipes, and from the rusted-over sprinkler head set beside the spinning light.

Along the back wall lay pieces of torn-apart machinery. Several empty egg-shaped devices, large enough for a fully grown changeling to fit comfortably inside, had been partially disassembled. Their remains lay discarded in a heap, cables twisted like electronic entrails. What really caught my attention, however, was the faint green glow of a terminal.

It was the first good news I had received since waking up, and I stumbled my way over to it, doing my best to ignore the protests of my legs; my muscles ached from disuse.

The ache made me think. The magical energies within a pod slowed muscle atrophy, but did not eliminate it, and my legs were already feeling the strain. They trembled slightly as I moved, a burning fatigue spreading up through my shoulders. I had been in that chrysalis for a long time. But how long?

A glance at the pod I was hobbling past told me it was a very long time, indeed. Many months, at the very least. Years, most likely. My mind couldn’t help pondering the changeling within; how long had it taken for her to die, and then rot away to an empty shell?

I shuddered, which quickly devolved into another coughing fit. A bit more of the viscous fluid dribbled from my lips as I recovered. I did my best to ignore the growing ache in my gut, taking small comfort that the fluids within a changeling pod were not foul-tasting; it had little flavor, like water with a hint of salt and sugar mixed in.

Once I felt sturdy enough, I continued past the pods to the terminal. I needed answers. Information is a powerful weapon, and right then, I was disarmed. I needed to fix that. I needed to arm myself with knowledge so I could begin to figure out what to do next.

And to tell the truth, I could desperately use \textit{something} to distract me from the death that surrounded me. Something, hopefully, that might give some sense that things could, somehow, be made right.

I faced the terminal screen, ignoring how the brief trip had already rendered me short of breath. Stress, I reasoned to myself, as I wiped away the thick layer of dust on the screen and hit a key. The terminal woke up with a faint whir and hum, the blank glow of the screen replaced by a simple message:

\terminal{
Welcome back, user \textit{CoolBugz.}
}

I choked out a weak laugh, halfway between humor at the login and crippling depression at finding amusement while surrounded by my dead hive-mates. I’m pretty sure my mind was not terribly stable at the time. I forced myself to swallow my grief and continue on.

With another press of the button, the welcome message disappeared, replaced by a logo in monochrome green text.

\terminal{
\textbf{Crystal Life Technologies}

For a better future

\textit{Resuming session...}
}

The terminal hummed for a few more moments as it worked, while I considered the name. It was vaguely familiar, though I couldn’t quite place it. The logo vanished before I could remember, presenting me with an alert message and a very rude bleep.

\terminal{
\textbf{Danger!}

S.A. pod \#4 (Experimental) failure!

Biomed control system failure!

Lifesign monitoring failure!

Check occupant immediately!
}

A quick count placed my chrysalis as the fourth from the left. At least there was some small comfort that the computer was concerned with my well-being, for what little that was worth.

I pressed a key to continue, and the warning was replaced by a new one.

\terminal{
\textbf{Warning!}

Primary power system failure.

Emergency power systems activated.

Containment locks released.

All personnel evacuate immediately!
}

Containment locks? I looked to the door of the room, standing open. It was one of those heavy, powered doors, the kind that probably weighed five hundred pounds and could stop a good-sized explosion. My gaze drifted to my Queen, resting just beside the door. Was that what was responsible for this? Had she woken, as I had, only to be trapped in here by this ‘containment lock’?

I swallowed around the returning lump in my throat, and hit the key again.

\terminal{
\textbf{Caution!}

External environment \textit{is not} safe.

External radiation at \textit{168\%} of safety threshold.

\textit{Minor} atmospheric contamination detected.

\textit{Moderate} water contamination detected.

Environmental seal compromised.

Structural breach detected.

Environmental isolation can not be established.

All personnel evacuate immediately.
}

I was trembling again. This was not the good news I was hoping for.

A final press of the key cleared away the last of the warning messages, returning the terminal to its standard interface. I ignored the message that opened automatically to look for the time-and-date display. I stared at it in confusion until I realized that there was some sort of error. The time-and-date display was showing nonsense. For some reason, the terminal thought it was a couple centuries in the future.

A little disheartened by the lack of sensible information, I read over the displayed message. It was titled “Daily Report”, and consisted of an extensive list of technical problems. The first entries were the condition of the cocoons, or “S.A. pods, (Experimental)” as it titled them. Nine reported complete failures, with the last reporting several warnings of degrading systems. Other warnings noted problems for the rest of what seemed to be a sizable facility, such as a pressure failure in the fire suppression system, flooding in “level two”, complete failure of the water processing and pumping system, and the failure of two of three air intake and purification systems.

The only thing of interest I caught in the long list of failures was that three of the eight spark generators had apparently still been working when the daily report was generated. Judging by the warning I had received on starting up the terminal, they had all failed simultaneously today.

At the bottom of the message, the clean and formatted text abruptly changed.

\terminal{
***Bugz***

***WakeUpCall***

EnvExternalRad1.68

EnvExternalAirTox0.63

EnvExternalWatTox3.32

EnvExternalAirTempTRUE

EnvExternalAirLiveTRUE

WakeUpCallFALSE
}

Whatever it was, it looked like something crudely hacked onto the end of the existing report. I noted right away that “EnvExternalRad” matched the warning message I had received about external radiation levels. Whatever this was, it seemed to be making note of what conditions were like outside, and if I had to guess, deciding whether or not to wake us up based on what it found.

Or in short, it said the world outside was poison, and I shouldn’t be awake yet.

\textit{Really not what I wanted to hear right now}, I had thought. I’m pretty sure now that the fact I was silently talking to the terminal within my own head was a sign I was uncomfortably close to simply losing it.

An option flashed near the top of the screen, titled “New unread daily reports.” It was a moment of hope for me; if this terminal had been generating and storing daily reports ever since we had entered our chrysalises, the number of messages it contained should tell me how long I was gone!

I hit the option, and the screen changed to a list of reports. The “new messages” number exploded.

I was completely unsurprised when it almost instantly hit three digits, and only a little disappointed when it added a fourth digit, barely a second later. I sighed, slumping a little, but I was hardly surprised. A few months was beyond optimistic. A few years was much more reasonable, given the state of the room and the… decay.

The number kept going.

Horror started to dig at my gut as the number climbed higher and higher, refusing to stop. Years rolled by before my eyes.

It hit five digits, and my hindlegs gave out. I sat down, staring numbly. My brain did some quick math, completely without my bidding. Roughly thirty years. And still, it counted higher.

I continued to stare, unmoving, transfixed by the impossible horror of what I was seeing.

Finally, it stopped.

\terminal{
You have \textit{73,741} unread daily reports.
}

Again my brain did some quick math. If that number was correct—and I really didn’t want to believe it!—that was roughly two hundred years. Part of me suggested that it was wrong. The terminal had the wrong date, maybe it made up a bunch of daily reports for a bunch of days that didn’t actually happen!

Even at the time, I knew I was grasping at straws, but it was either that or drown in the realization that \textit{everyling} was dead. My hive was \textit{gone}.

I trembled, leaning against the terminal, and sobbed. Tears flowed down my snout as I shook, on the verge of giving up, but I let out a weak, angry cry as I thumped a hoof against the terminal’s housing. I was alive! My Queen had gone through all this effort for us, and I was alive! If I was alive, others could be, too! I couldn’t just lay there, crying like some spoiled filly as I threw away my Queen’s gift!

I pushed myself upright, doing my best to control my sobbing, and wiped at my eyes again, to no better effect than the previous attempt. I hit the “back” button with more force than necessary, eyes narrowing as I did my best to channel all my pain into determination. I glared at the terminal like it was an enemy I needed to extract information from.

A quick search turned up a socket, and I plugged in the data-store. The terminal hummed again, punctuated by the occasional whine. It kept working, and working, and working. After half a minute, it finally displayed a message.

\terminal{
\textbf{Error:} could not read external device.

File system corrupted.
}

The fire that had been growing in me died. I managed to hold on for a couple seconds against the wave of depression, but I finally buckled with a loud sob. I slid down against the terminal until I lay curled up before it, trembling and crying.

My Queen’s final gift, and it was denied to me.

My sobbing mixed with coughing as I spat up a bit more fluids, my chest aching as the muscles complained. A particularly bad coughing fit left me groaning, clutching my gut. At least it gave me something else to occupy my mind. The desperate sadness was shunted to the background as I simply focused on breathing.

I think that’s the moment that I really returned. The moment my brain finally woke and came up to speed. The grogginess from my extended rest still left me in a faint haze, but as I focused on the burning ache in my chest and belly, my mind started to tear apart the situation, evaluating it. My crying died away as I lay there, thinking.

\textit{I am on my own again}, I thought. That wasn’t so bad, even if the reason for it was horrific. I was an Infiltrator. I worked on my own regularly, often for months at a time. I had been trained to work independently, to approach and deal with problems without the guarantee of assistance. Yes, I didn’t know if anything of my hive had survived, especially if it had really been as long as the terminal claimed it had been. But I was equipped to work on my own in uncertain situations.

If anything of my hive \textit{had} survived, I could find it.

My Queen had brought us here for a purpose. She brought us here to preserve the hive.

I had a mission.

Slowly, I pushed myself back up onto wobbly legs. I blinked away the last of the tears, refocusing on the terminal. After a deep, steadying breath, I reached out, hitting a key to back out to the main menu once again, and started searching through the rest of the terminal.

The search turned up little. There was no personal correspondence, though there were lots of research notes that flew over my head. The only thing I got out of that was the mention of “suspended animation”. It led me to finally recognize where I recognized “Crystal Life Technologies” from. C.L.T. was some small-bit company that had managed to have dealings with five of the six ministries of the Equestrian government, thanks to its research in cryonics and suspended animation. It wasn’t a big player, not even close, but those connections drew enough interest to be one of the companies we kept tabs on.

Unfortunately, that recognition didn’t help me any, now.

The terminal seemed to be used solely for research and operation of the “experimental” suspended animation pods. There was nothing useful to me. It didn’t even have a map.

I sighed as I turned away from it. Looking back at the rest of the room prompted a moment of hesitation and sadness, but it faded quickly. I had cried too much already. Now it was time to act.

\textit{Step one: survive.}

Considering my situation, I knew that might be a bit of a challenge all on its own. I had hardly done anything since waking, and already my body was exhausted. My magic was almost depleted, and I was sure I’d need plenty of food and water soon. I had no knowledge of what was out there, except for the vague environmental reports the terminal had provided. Sadly, of all the technical jargon and analysis the terminal had contained, none of it gave even a vague idea of what those environmental measurements actually meant. I’d have to dig around in the actual code, and I didn’t have the tools for that.

Not that I really needed fine details to get the idea that “minor atmospheric contamination” was probably a bad thing.

I needed supplies, and I needed a way to carry them. I retrieved the data-store, hoping that I might be able to extract the data it had held with the proper tools. The love crystals were definitely coming with me, with the hope that I might someday acquire enough love that I might need to store some of it—and if I’m entirely honest, in part because they had come from my Queen. And, since I didn’t want to spend more of my limited magic than absolutely necessary by levitating them all the time, I needed something to carry it all in.

A quick scan of the room turned up one of those ubiquitous wall-mounted Ministry of Peace medical boxes, the paint mostly peeled away. Half of the box was badly corroded by the years of water trickling down the wall and over it, but it would serve my purposes. It took a couple good jerks to free it from the rusted mounting brackets, and I set it down on a table and opened the box.

It had once held a healing potion, but it had long ago broken, the liquid pooling and ruining most of the contents. Only a single bandage had survived. I set that aside and dumped out the rest of the contents. Glass shards clattered across the table, along with the rotten remains of a couple more bandages, a pair of corroded injectors, their labels decayed away and their contents likely ruined, and, of all things, a badly rusted Equestrian Army service pistol.

I stared at the pistol for a couple seconds, unable to help thinking that the M.o.P. would have been quite upset to find such a thing stored within one of their \textit{medical} kits.

Still, a pistol might come in handy. They weren’t useful for infiltration and impersonation—in fact, weapons could be a major liability!—but they could be useful in case of emergencies. Every Infiltrator received basic weapons training because of that. Some, those trained up for more \textit{direct} action, received even more, but I was not one of those operatives. Still, I had practiced on a model just like that pistol, albeit in much better condition.

Unfortunately, the condition of this pistol was far too poor. A few quick tugs revealed that the slide had rusted firmly in place, and I expected the internals had fared no better. It would probably need serious work to ever fire again, and I would hardly know where to start. I might be able to cludge my way into getting it working if I had a full tool shop and many hours to poke and prod at it, but that was time I simply didn’t have.

Leaving the gun behind, I tapped the medical box against the edge of the table until it was more or less dry, then placed the crystals and data-recorder within it, with the latter carefully wrapped in the intact bandage.

I closed the box, picked it up in my mouth, and looked around the room. My eyes lingered on each of my fallen sisters before stopping with my Queen. I knew I couldn’t leave them. I had to go, but I couldn’t just leave them, lying there, abandoned.

I stepped out the door and into the hallway. It was long and dark, illuminated at one end by another spinning emergency light. The entire area screamed utilitarianism and practicality: all bare concrete, with a simple metal grate along the floor for traction. Pipes ran along the ceiling under dead lights. One pipe had broken in half, jagged ends fallen to the floor. Whatever they had carried had stopped flowing long ago. At least it was dry, unlike most of the floor in the room I had awakened in. The regularly spaced drains seemed to still be doing their job.

I set the medical box beside the door, and began the slow process of gathering the remains of my fallen sisters. Their husks were disturbingly light, producing hollow drumbeats any time a part of the empty exoskeletons bumped against anything. I winced every single time.

The last two were the trickiest, still contained within their pods. I looked at the first, floating almost ghost-like in the murky fluids. The rotating light gave the sight a spectacularly creepy vibe, occasionally lighting the silhouetted form within.

With no knife or other cutting implement, I knew it was going to be messy, but I wasn’t going to let that prevent me from giving my sister this one final honor.

So, lacking a more obvious solution, I stabbed my horn through the membrane, just as I had done to free myself.

It was not, in retrospect, one of my brighter decisions.

The chrysalis burst around my horn with a spray of fetid muck. I recoiled in surprise even before the smell hit me; the overpowering combination of rot and bile sent me staggering blindly away, eyes screwed shut. I stumbled into and scrambled over several pieces of machinery as I fled the horrible geyser of awfulness, until I finally collapsed in the back corner of the room, heaving.

The first few heaves vomited up what fluids I had ingested from my own chrysalis, but they didn’t stop there. My gut clenched again and again, my muscles burning as my stomach tried to purge itself further. I used my hooves in an attempt to wipe the vile fluid from my face, groaning between heaves, and doing my best to ignore the few tiny bits of squishy \textit{stuff} that I flung away.

I could barely stand by the time my gut finally relented. I only made it a few feet, to a shallow puddle of murky water, and sank down to the ground again. Every muscle hurt, protesting even at the mild activity of splashing some of the water over my face. I’d never wanted a bath as badly in my entire life.

Once I had finally washed away enough of the mess to no longer feel \textit{completely} revolted, I heaved myself up to sit back against the row of gutted machinery. I was panting, a hoof draped across my snout in an attempt to block out some of the vile scent filling the room. My body practically screamed at me. I would have gladly lain down and slept, but I knew that wasn’t an option. I likely had a long couple of days ahead of me if I was going to survive, and every moment I spent idle meant another moment closer to death.

The deflated chrysalis taunted me from across the room, the sodden and decayed husk tangled up in its dripping remains. Reluctantly, I rose back to unstable hooves, and started slowly rooting among the scattered parts. One of the egg-shaped pods in the back eventually gave me what I was looking for. Part of its curved, white outer shell had been broken away when it had been disassembled, leaving a large chunk of plastic with a jagged edge. I grabbed it in my mouth and got to work.

Cutting the final strips of the pod away to free my deceased sister was awkward with my crude blade, but it was over quickly enough. I dragged the husk over to join the others, ignoring how slimy the shell felt under my hooves, and then it was time to deal with the final pod.

I gripped the plastic shard in my hooves, giving a couple beats of my wings as I balanced on my hindlegs, and jammed it into the pod. It took a couple jabs, but the chrysalis tore open just like the last, though without me getting a faceful of its contents. I drew back and retreated to the hall, taking the opportunity to get some relatively fresh air and let my aching muscles rest while the cocoon drained.

It was even harder forcing myself to rise again, but I knew it was almost done. A little bit of crude sawing and a few moments of dragging, and I had finally gathered all of my fallen hive-mates together. My Queen lay there, still impressive even in death, her daughters clustered close around her. I suppose it was some small comfort that I didn’t recognize any of my sisters. A few looked vaguely familiar, but that was it. Perhaps I had interacted with a few of them in a professional sense, but I didn’t see any of those I had been close to. It left some tiny hope that they had survived, as I had.

Eventually, I lowered my head, pushing a little of my dwindling magic into my horn. I knew every bit of magic I spent was a little less to sustain me, but I had to do this one last thing. They deserved that much.

A green flame burst into being beneath my Queen’s chest and rapidly spread. The dry, empty husks caught quickly; the more sodden remains took longer, but soon they too succumbed to the flame. The room flickered and danced in the light of the green fire.

I rested, watching the flames transform their bodies one final time. Eventually those flames faded, guttered, and died, leaving nothing but ash in their wake.

I sighed as the light died away. There was nothing left to tie me to that place. My duty to my fallen kin fulfilled, I stood, turned, and left.

Retrieving my commandeered medical box, I set off to find my way out of the facility. The hall only went one way, the other blocked off by a door that had only opened a couple inches before wedging firmly in place. I had to hope that wasn’t the way out, as I was certain I couldn’t move it even with a full reservoir of magic.

Rooms branched off from the hall, pricking at my curiosity, but I didn’t have the time or energy to indulge more than a quick glance. Most were uninteresting. There was a cleaning closet, restrooms, and even some quarters. A couple rooms looked much like the one I had woken up in, though they were in various states of completion. One held a dozen of those egg-shaped pods I had seen before, though only half of them were assembled and hooked up. Those rooms that were not damp from leaking water were caked in dust.

Eventually the hall opened into a larger room, looking much like a lobby. The far side had partially collapsed, the mess of rock and rubble corroborating my impression that I was underground; unless this place had been built to contain a small megaspell or had secret passages riddled throughout it, there was no way an above-ground structure would have so much dead space between rooms.

I could finally make out the voice, repeating an endless loop from a speaker in the wall. It was a mare’s voice, speaking calmly, though the speaker warbled badly.

\leavevmode\llap{“}Attention all personnel: please exit the facility and proceed to designated evacuation locations.”

Opposite of the partial cave-in were wide doors leading into a vertical shaft. The stairs that had once occupied the space had collapsed, leaving a heap of rubble partially submerged in the water below. Above, a doorway opened out into the void of the collapsed stairway.

A short flight landed me on the upper landing, a warmth spreading through my chest as my wing muscles began to join all the others crying out at my abuse. I ignored them and continued on down the hall I found myself in.

The first room I glanced in was filled with pony skeletons. At least twenty of them had been heaped up in the small storage room, laid two or three deep. I quickly continued on.

The next room held a completely different form of desolation. The room, about the same size as the one I had woken in, had been destroyed in a fire. The walls were scorched black, and the multitude of storage boxes within had been destroyed. The room would have been a treasure trove if it had been intact. One crate had held at least twenty medical boxes, now warped and ruined from the flames. Another held the remains of a couple dozen rifles. Beside that were the twisted remains of a metal box that had been torn apart from the inside; I was guessing that had been ammunition for the rifles, or maybe explosives, and I wondered if it had succumbed to the fire or been its source. Along the side wall hung six sets of armored security barding, complete with riot-style helmets, now all tattered rags, loose-hanging plates, and warped plastic.

If I had the time and the skill, I might have been able to cobble some things together from the remains, but I had neither of those things.

One box sat outside the room, and as such, had been spared the flames. I opened the lid, and was greeted with the sight of dozens of military-style rations. I quickly opened my medical box, loading it until I could barely close the lid over them. Then I tore open another ration, scarfing down the contents. My stomach, still aching from the earlier exertion, felt slightly more comfortable.

Tossing aside the wrappers, I continued on. There were more rooms, and more signs of decay. There were even a spattering of bullet holes in one of the walls, and another bore a scorched crater left by some sort of magical energy weapon.

The further I went, the more familiar it started to feel. It was vague at first, walking by a small lounge area that I felt I had seen before, in better lighting than the flickering emergency lights that dotted the facility. A look back down the hallway I was walking gave a sense of deja vu, even with the lights hanging unevenly and the water dripping from broken pipes. When I stepped out into the lobby at the end of the hall, I could swear I’d stood there before.

A faint breeze of fresh air met me, drawing me toward the glow coming in through the open door. I stepped into the doorway, and I remembered.

{\br}%
I remembered the end of the world.

I had been flying.

I left Appleloosa that morning. I had woken to an encrypted message I had never expected to receive. It was an emergency recall, and not just any. It was the most extreme category; drop everything immediately, abandon all resources, and flee the country as fast as you possibly can. It was the kind of message we dreaded getting, the sort of warning one would get moments before the Ministry of Morale burst in.

And it was being sent out to every single Infiltrator in the world.

I fled. I didn’t call into the shipping department that I worked at to excuse my impending absence. I didn’t grab anything to take with me. I simply changed into a pegasus, ran out the back door, and flew as fast as I could. The message had given me a rendezvous point, only a couple hours away.

I was flying when a flash lit up the sky. I continued flying as more lights flashed behind me, and then off to the sides. Then another flash, but in front of me, far away enough that it had to be beyond the border of Equestria. Far off to the south, in the Badlands. There were no pony settlements there. No zebras. Only changelings.

As more flashes lit the horizon, I dove toward my destination.

{\br}%
I stepped out into a compound that I had remembered seeing from the air. The nearby shelter had collapsed, smashing the pair of skywagons that had been parked beneath it. The guard shack, looking over the broken gate in the chain-link fence, was missing its door and windows, its paint having mostly peeled away. Only the train tracks looked more or less intact, leading away toward the main line in the valley beyond. A worn sign stood beside the guard shack, faded letters declaring this to be, “Crystal Life Technologies Experimental Site Alpha.”

Behind me, the small concrete entryway poked out of the side of a hill, just as I remembered.

I didn’t recognize the huge \textit{thing} lying just beyond the fence, but judging from the way it smoked and sizzled, I was guessing it was new. It was as large as the skywagon shelter, with a smooth, slightly curved surface on one side, and ragged edges and bits of metal on the other. It looked like it had once been part of a large structure of some sort, though the deep furrow gouged through the earth suggested it had traveled a good ways from there. I couldn’t imagine the amount of force it had required to throw something so large.

Looking away from the strange debris, I was surprised at just how bleak the place looked. Sure, much of southern Equestria was fairly barren and dry, but there was plenty of life to be found if you knew where to look. I saw nothing like that, there. I saw only a few living plants, and they were brown and sickly, barely clinging to life. A few dead trees dotted nearby ridges, roots partially exposed in the crumbling, dry soil. Even the air, while fresh in comparison to the musty scent inside the facility, seemed stale and bland.

At least it was sunny and pleasantly warm, though the dark clouds that filled half the sky, dominating the horizon, threatened to change that. The more distant lands were darkened by their shadows, looking dull and oppressive. The clouds seemed particularly unruly and chaotic, as if the pegasi arranging them couldn’t be bothered to care, and had simply tossed them wherever.

I set down my box, taking advantage of a deep puddle—only slightly muddy—to finally get an acceptable pass at bathing. Being plastered with dried ichor and smelling of rot would probably not help me survive the next few days.

A couple minutes later I emerged, feeling a little better about myself.

I reclaimed my medical box, and looked around once more. I had a pretty good idea of what direction was what, if my hazy memory of this place was right. I couldn’t remember much of what happened after I arrived—mostly just my Queen, in all her glory!—but I was fairly certain I had come flying in over the skywagon shelter.

So that just left the question of what direction to go. Ultimately, I wanted to go south; if there were any sign of my hive’s survival, the hive itself would be the first place to look. Unfortunately, pony settlements were increasingly sparse the further south you traveled. Then again, who knew what had changed while I was asleep. Especially, though I still wasn’t ready to accept it, if it had been two centuries.

North would lead me deeper into Equestria. Sure, I’d seen the flashes in the distance, and I’d seen the predictions of the effects of a full megaspell exchange; major population centers would have been reduced to ash, but two hundred years was a long time to recover—though my mind momentarily rebelled at using such an absurd figure as a \textit{good} thing. In any case, there would almost certainly be more ponies in that direction.

But it also took me further from my hive.

Eventually, I settled on a third option: east. Dodge City lay somewhere in that direction, and there were many smaller towns that likely would have escaped targeting. It would be as close to my hive as I could reasonably expect to find a good number of ponies.

It seemed to be an ideal compromise.

I pulled out a cable from the wreckage of a skywagon, running it through the brackets on the back of my medical box and tying it into a loop. It was difficult to do without magic, but I’d spent a lot of time imitating an earth pony. I slipped the loop around my neck, spread my wings, and took to the air.

I only managed to make the next ridge before I had to land, my wings aching horribly, but that part at least went to plan. I had no idea what was out there, so the plan was to go slow and careful. I landed just before the ridge and creeped up to peer over, instead of just flying blindly over it. It took longer, but even with the frequent pauses to scout ahead and rest my wings, I figured it would take me less than a day to get to my destination. Probably half that, even.

After a couple minutes' rest, I was ready to move on to the next rise, perhaps half a mile away across a shallow valley. My wings ached even worse as I landed again, but it was progress.

My rests grew longer with each stop, quickly seeding doubts as to my expected rate of travel.

Then, on the fifth stop, I saw movement. The railroad tracks had snaked their way through the sparse terrain, and just half a mile away to the south, it curved around a small hill. I had peeked over the ridge just in time to see the hindquarters of some quadruped disappearing around the corner, following the tracks.

I waited just long enough to make sure nobody came back, then took off again, flying low over the ground. I ignored the protests of my wings and the slowly spreading fire of overworked muscles, limping the final stretch before practically collapsing on the slope of the small hill. I crept up, slowly peering over it.

I was rewarded with the wonderful sight of ponies! Five of them, two mares and three stallions, along with a pair of cattle. They were armed, too. The ponies, that is, not the cattle, who simply carried large packs. The ponies had various small arms, and two of the stallions wore light, armored barding. They didn’t look like any military group. They seemed disorganized and irregularly equipped. Then again, maybe they were deserters? Or were they simply armed travelers? I had no way to tell.

Whatever the case, it was all secondary in importance. I had found ponies!

Now, I suppose I should make it clear: I don’t hate ponies. I know some ponies got their entire impression of changelings from Queen Chrysalis and her hive, but that would be a very incomplete picture. Me, I actually \textit{like} ponies. Sure, they can be skittish and scared of things they don’t understand, and yes, they did not handle the tremendous stress of the decades-long war they found themselves in very well, but they’re basically decent beings. Few creatures held so much love, even if their hardships had made them a bit paranoid about sharing it.

My point is, my hive didn’t view ponies as being just prey, much less enemies. If anything, they’re more like… valued livestock. I know, that probably sounds horrible, but it’s the best comparison I can think of. I won’t try to spin it, to portray us as some sort of silent, benevolent protectors. We had selfish reasons for our actions. Ponies were good food, and we wanted to keep that food safe. That was our primary focus.

But when you spend much of your life living among ponies, forming friendships in order to get the love your hive needs, it’s hard to not start liking the ponies that feed you. They could never compete with the bond I shared with my sisters, much less my Queen, but I still enjoyed their company.

So finding a small group of them here offered me not only food, but the chance of some small degree of comfort.

That was getting a bit ahead of myself, though. Before I could do any of that, I had to gain their acceptance. To do that, I wanted to get a better idea of what I might be walking into.

I hung back, keeping myself as hidden as possible as I watched them. They were following the tracks, and while the cattle merely plodded on without concern, the ponies kept looking to the sky. One mare, the only unicorn in the group, had a huge smile on her face as she looked around the sky. She stumbled a few times, not watching where her hooves were going, but each time her gaze turned upwards again.

The other mare and the unarmored stallion also looked quite pleased, walking close side-by-side and quietly talking with each other. I was guessing they were a couple.

As for the other two stallions, one looked to be as unconcerned as the cattle, though he wore a pleased expression. The other stallion was the most severe of the group, his eyes scanning the horizon, looking for threats. Yet even his gaze was drawn upward on occasion, the tension in his stance fading for a few moments. Then his eyes dropped, and he returned to looking around.

I drew back just a bit to hide myself as I looked up, to see if I could find what was drawing their attention. I didn’t. There wasn’t anything up there but the sky.

I returned to observing them.

The next two hours were pretty repetitive. I watched while they walked. Once they were out of sight, I flew up to the next piece of cover to observe again. Ten minutes of watching, a minute of flying, repeat. I didn’t mind the pace. It gave my poor wings time to recover between flights. By the time they stopped, with the sky steadily darkening, I was actually feeling halfway rested.

The ponies pulled off beside the tracks, unpacking bedrolls. Finally seeing the cattle from the front as they turned, I was surprised to see that one of them had two heads, and the other, while having only one, looked badly malformed.

Soon the ponies had a small campfire going. The unarmored stallion tossed ingredients into a cooking pot while the others relaxed. The severe-looking stallion kept a wary eye out even as he lay back against a rock, a crude-looking rifle set in easy reach.

I got a bit better look at the small arsenal they carried. The other armored stallion had a similar rifle slung on his back, while the couple had holstered pistols. The unicorn mare had a very long and narrow rifle, cradled in her forelegs as she lay back. She was still smiling.

The amount of weapons on display was a little concerning. If they carried weapons, that must mean there was something dangerous out there. My first thought was zebras, but it seemed a little silly to think that the war would still be going on after so much time. Maybe it was dangerous wildlife? Sure, I’d seen hardly any signs of life, and most of that was just desert brush, but that didn’t mean \textit{nothing} lived out here. There always seemed to be some horrible monster wandering into Equestria and causing drama.

With that thought, I was even more thankful to have run across ponies so soon. If there was something dangerous out there, I didn’t want to run into it on my own. I found myself glancing backwards to ensure nothing was sneaking up on me, and becoming very aware that the sunlight was quickly fading.

I would have liked to observe longer, but I figured they’d likely be going to sleep soon. Considering the vigilance the one pony displayed, I expected they would have someone keeping an eye out, making it hard to sneak up even in the dark. And, to be honest, I was thinking that I really didn’t want to wait out there, on my own, for whatever had made them so wary.

I silently retreated back down the slope, getting well out of sight. I didn’t want them to see the flash of my magic.

Once I was far enough away, hidden in a small depression beside the tracks, I called up a little bit of magic. The green flames washed over me, replacing black chitin with a smooth gray coat. I shook my head, fluffing out my new silver mane, long and sleek, and flicked my new tail. Finally, I stood up, looking over myself to ensure I had gotten the form correct. I hardly had to worry, though; it was the same form I had worn for a few years in Appleloosa.

It seemed an ideal choice. With no pegasi and only a single unicorn, an earth pony form should fit right in, and the fairly even split of sexes gave me no reason to not remain female. The cutie mark of a closed scroll was amazingly versatile; given just a bit of creativity, I could make it mean whatever I wanted. I could even use the same name: Whisper Winds.

Now, I know re-using disguises is generally a bad idea, but all things considered, I really doubted I would run into anyone who might recognize me. For that matter, I’m sure some might think that simply tacking “Winds” onto the end of my real name was a bad idea, but it’s not like anyone had a database of changeling names to check it against. As long as I didn’t re-use it for a different disguise, it was just fine, and it can be very beneficial to have a name you’re used to hearing.

With one final check of my new form, I returned to the ridge. Transformation was an easy spell, but I could feel how my hunger grew at the small expenditure. The sooner I could make some “friends”, the better.

I remember feeling a bit of irritation that my new earth pony muscles should feel just as worn out as my natural ones, but that’s just how shapeshifting goes. A proper meal would fix that.

Settling in and carefully peeking out once again, I saw nothing had changed during my short absence. They were all settled in and talking.

I braced myself, mentally going over several quick “facts” for an improvised backstory. Then I put on a pleasant smile, stood, and began to walk toward them.

They didn’t notice me, at first. The only one who seemed focused on looking outside of their small circle was facing away from me. The others were too preoccupied with their conversation. Soon I was close enough to make out what was being said.

\leavevmode\llap{“}...But seriously, rainbows!” the smiling mare said, looking up again to the darkening sky. “I’ve never even seen a rainbow before.”

\leavevmode\llap{“}We know,” the vigilant stallion said with a resigned tone. “You’ve only told us like fifty—”

He had turned to look squarely at the mare, which brought me into his edge of his vision. His reaction speed was impressive. He immediately bit down on the grip of his rifle as he rolled onto his hooves. I staggered to a stop, suddenly looking down a frightfully large barrel.

His voice came out harsh and menacing, his eyes narrowing at me. “Don’t you fucking \textit{move}.”

It was at that moment I began to worry that I had made a terrible mistake.